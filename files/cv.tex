\documentclass[letterpaper, 10pt]{article}

\usepackage[left=.5in, top=.25in, total={7.5in, 10.5in}, includeheadfoot]{geometry}

\usepackage{booktabs}

\usepackage{tabularx}

\usepackage{array}

% Custom command

\newcommand{\headings}[1]{\section*{#1} \hrule \vspace{10pt}}

\title{CV}
\author{bhowes }
\date{December 2017}

\begin{document}
\begin{center}
{\Large \textbf{Bolton Howes}}\\
PhD Canddiate\\
Department of Geosciences, Princeton University\\
Guyot Hall, Princeton, NJ 08544\\
(720)-937-9891, bhowes@princeton.edu\\
\end{center}

\headings{Education}
\begin{center}
	\begin{tabularx}{\textwidth}{>{\itshape\arraybackslash}l X}
		2017 \rightarrow & \textbf{Princeton University}, Princeton, NJ \newline Ph.D., Geosciences \newline Thesis: The sedimentary and geochemical record of climate change at the Cretaceous--Paleogene Boundary  
		\newline Advisor: Adam Maloof \\
		2015--2017 & \textbf{University of Georgia}, Athens, GA \newline M.Sc., Geology \newline Thesis: The application of sequence stratigraphic principles to constrain the onset of thrust load-induced subsidence: the Jurassic Twin Creek Limestone of Southwest Wyoming
		\newline Advisor: Steven Holland\\
		2011--2015 & \textbf{Macalester College}, St. Paul, MN \newline B.A., Geology \newline Thesis: Characterization of a regionally significant terrestrial bounding surface in the Upper Cretaceous Two Medicine Formation, northwestern Montana \newline Advisor: Raymond Rogers\\
	\end{tabularx}
\end{center}

\headings{Publications and Conference Abstracts}
\begin{center}
	\begin{tabularx}{\textwidth}{>{\itshape\arraybackslash}l X}
        2014 & \textbf{Bolton Howes} and Raymond Rogers. Revisiting a regionally significant terrestrial bounding surface in the Upper Cretaceous (Campanian) Two Medicine Formation, northwestern Montana. GSA Annual Meeting in Vancouver, British Columbia.\\
        
        2018 & \textbf{Bolton Howes}, Akshay Mehra, and Adam Maloof.  Three-Dimensional Reconstructions of Holocene and Neoproterozoic Oolites to Measure Porosity, Permeability, and Volume-Shape Evolution of Ooids\\
        
        2021 & Paul Hoffman, Galen Halverson, Daniel Schrag, John Higgins, ... \textbf{Bolton Howes}. Snowballs in Africa: sectioning a long-lived Neoproterozoic carbonate platform and its bathyal foreslope (NW Namibia). Earth-Science Reviews\\
        
        2021 & \textbf{Bolton Howes}, Akshay Mehra, and Adam Maloof. Three-Dimensional Morphometry of Ooids in Oolites: a new tool for more accurate and precise paleoenvironmental interpretation. Journal of Geophysical Research: Earth Surface.\\ 
        
        2022 & Emily Geyman, Ziman Wu, Matthew Nadeau, Stacey Edmonsond, Andrew Turner, Sam Purkis, \textbf{Bolton Howes}, Blake Dyer, Anne-Sophie Ahm, Nan Yao. and Curtis Deutsch. The origin of carbonate mud and implications for global climate. PNAS.\\
        
        2022 & Adam Maloof, Ryan Manzuk, Emily Geyman, Akshay Mehra, Jaap Kaandorp, Mark Webster, Stacey Edmonsond, \textbf{Bolton Howes}, and Cedric Hagen. From modern analogs to three dimensions: Lessons learned for interpretting the stratigraphic record of the Proterozoic--Phanerozoic transition. GSA Annual Meeting in Denver, Colorado.\\
        
        2022 & Akshay Mehra, \textbf{Bolton Howes}, Ryan Manzuk, Alec Spatzier, Bradley Samuels, Adam Maloof. A novel technique for producing three-dimensional data using serial sectioning and semi-automatic image classification. Microscopy and Microanalysis.
	\end{tabularx}
\end{center}


\headings{Leadership and Service}
\begin{center}
	\begin{tabularx}{\textwidth}{>{\itshape\arraybackslash}l X}
	    2018--2021 & Residential Graduate Student, Princeton University\\
		2016--2017 & Student Volunteer, Geological Society of America\\
		2015 & Leadership Committee of Macalester College Fellowship of Christian Athletes\\ 
		2012--2013 & Resident Assistant, Macalester College\\
		2011--2014 & Captain of Macalester College Football Team\\
	\end{tabularx}
\end{center}

\clearpage
\headings{Teaching Experience}
%\textit{All of the listed courses involve a significant fieldwork component consisting of a week-long trip to make observations for research projects (the locations of the fieldwork are in parentheses). As part of my teaching duties, I was responsible for preparing supplies and logistics and helping students with both data collection – especially with drones, differential GPS, and surveying equipment – and data analysis.}
\begin{center}
	\begin{tabularx}{\textwidth}{>{\itshape\arraybackslash}l l X}
	    Spring 2021 & Teaching Assistant & \textbf{ENV 354-GEO 368} Climate and Weather: Order in the Chaos\\
	    Fall 2019 & Teaching Assistant & \textbf{FRS 161:} Earth: Crops, Culture, and Climate (in Italy)\\
		Fall 2018 & Teaching Assistant & \textbf{GEOL 201:} Measuring Climate Change: Data Analysis and Science Writing\\
		2016--2017 & Geology Tutor & University of Georgia Student-Athlete Academic Center\\
		Spring 2015 & Teaching Assistant & \textbf{GEOL 165}: History and Evolution of the Earth\\
		Fall 2014 & Teaching Assistant & \textbf{GEOL 300}: Paleobiology\\
	\end{tabularx}
\end{center}

%\headings{Journal Articles in Review}
%\begin{enumerate}
%	\item{\textbf{Mehra, A.} and Maloof, A.C. 2017, A multiscale approach reveals that Cloudina aggregates are detritus and not \textit{in situ} reef constructions. \textit{Proceedings of the National Academy of Sciences of the United States of America, in review.}}
%\end{enumerate}
\headings{Awards and Honors}
\begin{center}
	\begin{tabularx}{\textwidth}{>{\itshape\arraybackslash}l X}
		2017 & Masters Student of the Year, University of Georgia\\
		2015--2017 & Graduate Student Assistantship. University of Georgia\\
		2016 & Graduate Research Grant, Geological Society of America\\
		2016 & Watts-Wheeler Research Award, University of Georgia\\
		2015 & Henry Lepp Award for Dedication to Scientific Research, Macalester College\\

	\end{tabularx}
\end{center}
%\clearpage

\headings{Fieldwork Experience}
\begin{center}
	\begin{tabularx}{\textwidth}{>{\itshape\arraybackslash}l l l X}
	    2019 \& 2022 & Bolivia & 12 weeks & The K--Pg Boundary in lacustrine and fluvial settings\\
	    2021 & southern Utah & 2 weeks & Cyclicity in the Late Paleozoic Ice Age \\
	    2019 & Calabria, Italy & 1 week & Measuring climate change in olive orchards \\
	    2019 & Yukon, Canada & 5 weeks & Archeocyathid reefs and the Cambrian Explosion\\
	    2018 & Spain and Italy & 6 weeks & The K--Pg Boundary and slope sediments \\
	    2018 & southern Namibia & 4 weeks & Neoproterozoic and Paleozoic glacial deposits \\
		2018 & Western Australia & 1 week & Modern and recent carbonate platforms \\
		2018 & Nevada/Arizona & 3 weeks & Stratigraphy and Geochemistry of the late Paleozoic ice age \\
		2017 & northern Namibia & 6 weeks & Field assistant to Paul Hoffman: Stratigraphy and mapping of Snowball Earth deposits of the Otavi Group \\
		2016 & western Wyoming & 6 week & Sequence stratigraphy of the Jurassic Twin Creek Limestone\\
		2014 & NW Montana & 2 weeks & Sequence stratigraphy and stratigraphic paleobiology of the Cretaceous Two Medicine Formation\\
		2014 & Patagonia, Chile & 12 weeks & Restoration ecology and conservation biology in the Ayse\'n Region\\
		2013 & NW Wyoming & 5 weeks & Terrestrial Stratigraphy and paleobiology of the Cloverly Formation \\
		2014 & Tall Dhiban, Jordan & 6 weeks & Archeological excavation  Early Bronze Age through the 19th century \\
	\end{tabularx}
\end{center}


%\headings{Professional Experience}
%\begin{center}
%	\begin{tabularx}{\textwidth}{>{\itshape\arraybackslash}l X}
%		2011 -2013 & Situ Studio, Brooklyn, NY \newline Developed specifications for GIRI (Grinding, Imaging and Reconstruction Instrument), part of the new Digital Fossil Reconstruction Lab at Princeton University. Leveraged numerous architectural tools and methods to create visualizations and analysis for the Forensic Architecture project.
%	\end{tabularx}
%\end{center}


\headings{Continued Education}
\begin{center}
	\begin{tabularx}{\textwidth}{>{\itshape\arraybackslash}l X}
		2018 & Agouron Advanced Astrobiology and Geobiology Field School in central Namibia, Agouron Institute\\
		2016 & Reservoir Engineering, ExxonMobil\\
		2016 & Geochemistry of the Cretaceous Western Interior Seaway, Geological Society of America\\
		2015 & Sequence Stratigraphy for Graduate Students, ExxonMobil\\
	\end{tabularx}
\end{center}


\end{document}
